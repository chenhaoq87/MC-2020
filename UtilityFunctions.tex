\documentclass[]{article}
\usepackage{lmodern}
\usepackage{amssymb,amsmath}
\usepackage{ifxetex,ifluatex}
\usepackage{fixltx2e} % provides \textsubscript
\ifnum 0\ifxetex 1\fi\ifluatex 1\fi=0 % if pdftex
  \usepackage[T1]{fontenc}
  \usepackage[utf8]{inputenc}
\else % if luatex or xelatex
  \ifxetex
    \usepackage{mathspec}
  \else
    \usepackage{fontspec}
  \fi
  \defaultfontfeatures{Ligatures=TeX,Scale=MatchLowercase}
\fi
% use upquote if available, for straight quotes in verbatim environments
\IfFileExists{upquote.sty}{\usepackage{upquote}}{}
% use microtype if available
\IfFileExists{microtype.sty}{%
\usepackage{microtype}
\UseMicrotypeSet[protrusion]{basicmath} % disable protrusion for tt fonts
}{}
\usepackage[margin=1in]{geometry}
\usepackage{hyperref}
\hypersetup{unicode=true,
            pdftitle={UtilityFunctions},
            pdfauthor={Sarah I. Murphy},
            pdfborder={0 0 0},
            breaklinks=true}
\urlstyle{same}  % don't use monospace font for urls
\usepackage{color}
\usepackage{fancyvrb}
\newcommand{\VerbBar}{|}
\newcommand{\VERB}{\Verb[commandchars=\\\{\}]}
\DefineVerbatimEnvironment{Highlighting}{Verbatim}{commandchars=\\\{\}}
% Add ',fontsize=\small' for more characters per line
\usepackage{framed}
\definecolor{shadecolor}{RGB}{248,248,248}
\newenvironment{Shaded}{\begin{snugshade}}{\end{snugshade}}
\newcommand{\AlertTok}[1]{\textcolor[rgb]{0.94,0.16,0.16}{#1}}
\newcommand{\AnnotationTok}[1]{\textcolor[rgb]{0.56,0.35,0.01}{\textbf{\textit{#1}}}}
\newcommand{\AttributeTok}[1]{\textcolor[rgb]{0.77,0.63,0.00}{#1}}
\newcommand{\BaseNTok}[1]{\textcolor[rgb]{0.00,0.00,0.81}{#1}}
\newcommand{\BuiltInTok}[1]{#1}
\newcommand{\CharTok}[1]{\textcolor[rgb]{0.31,0.60,0.02}{#1}}
\newcommand{\CommentTok}[1]{\textcolor[rgb]{0.56,0.35,0.01}{\textit{#1}}}
\newcommand{\CommentVarTok}[1]{\textcolor[rgb]{0.56,0.35,0.01}{\textbf{\textit{#1}}}}
\newcommand{\ConstantTok}[1]{\textcolor[rgb]{0.00,0.00,0.00}{#1}}
\newcommand{\ControlFlowTok}[1]{\textcolor[rgb]{0.13,0.29,0.53}{\textbf{#1}}}
\newcommand{\DataTypeTok}[1]{\textcolor[rgb]{0.13,0.29,0.53}{#1}}
\newcommand{\DecValTok}[1]{\textcolor[rgb]{0.00,0.00,0.81}{#1}}
\newcommand{\DocumentationTok}[1]{\textcolor[rgb]{0.56,0.35,0.01}{\textbf{\textit{#1}}}}
\newcommand{\ErrorTok}[1]{\textcolor[rgb]{0.64,0.00,0.00}{\textbf{#1}}}
\newcommand{\ExtensionTok}[1]{#1}
\newcommand{\FloatTok}[1]{\textcolor[rgb]{0.00,0.00,0.81}{#1}}
\newcommand{\FunctionTok}[1]{\textcolor[rgb]{0.00,0.00,0.00}{#1}}
\newcommand{\ImportTok}[1]{#1}
\newcommand{\InformationTok}[1]{\textcolor[rgb]{0.56,0.35,0.01}{\textbf{\textit{#1}}}}
\newcommand{\KeywordTok}[1]{\textcolor[rgb]{0.13,0.29,0.53}{\textbf{#1}}}
\newcommand{\NormalTok}[1]{#1}
\newcommand{\OperatorTok}[1]{\textcolor[rgb]{0.81,0.36,0.00}{\textbf{#1}}}
\newcommand{\OtherTok}[1]{\textcolor[rgb]{0.56,0.35,0.01}{#1}}
\newcommand{\PreprocessorTok}[1]{\textcolor[rgb]{0.56,0.35,0.01}{\textit{#1}}}
\newcommand{\RegionMarkerTok}[1]{#1}
\newcommand{\SpecialCharTok}[1]{\textcolor[rgb]{0.00,0.00,0.00}{#1}}
\newcommand{\SpecialStringTok}[1]{\textcolor[rgb]{0.31,0.60,0.02}{#1}}
\newcommand{\StringTok}[1]{\textcolor[rgb]{0.31,0.60,0.02}{#1}}
\newcommand{\VariableTok}[1]{\textcolor[rgb]{0.00,0.00,0.00}{#1}}
\newcommand{\VerbatimStringTok}[1]{\textcolor[rgb]{0.31,0.60,0.02}{#1}}
\newcommand{\WarningTok}[1]{\textcolor[rgb]{0.56,0.35,0.01}{\textbf{\textit{#1}}}}
\usepackage{graphicx,grffile}
\makeatletter
\def\maxwidth{\ifdim\Gin@nat@width>\linewidth\linewidth\else\Gin@nat@width\fi}
\def\maxheight{\ifdim\Gin@nat@height>\textheight\textheight\else\Gin@nat@height\fi}
\makeatother
% Scale images if necessary, so that they will not overflow the page
% margins by default, and it is still possible to overwrite the defaults
% using explicit options in \includegraphics[width, height, ...]{}
\setkeys{Gin}{width=\maxwidth,height=\maxheight,keepaspectratio}
\IfFileExists{parskip.sty}{%
\usepackage{parskip}
}{% else
\setlength{\parindent}{0pt}
\setlength{\parskip}{6pt plus 2pt minus 1pt}
}
\setlength{\emergencystretch}{3em}  % prevent overfull lines
\providecommand{\tightlist}{%
  \setlength{\itemsep}{0pt}\setlength{\parskip}{0pt}}
\setcounter{secnumdepth}{0}
% Redefines (sub)paragraphs to behave more like sections
\ifx\paragraph\undefined\else
\let\oldparagraph\paragraph
\renewcommand{\paragraph}[1]{\oldparagraph{#1}\mbox{}}
\fi
\ifx\subparagraph\undefined\else
\let\oldsubparagraph\subparagraph
\renewcommand{\subparagraph}[1]{\oldsubparagraph{#1}\mbox{}}
\fi

%%% Use protect on footnotes to avoid problems with footnotes in titles
\let\rmarkdownfootnote\footnote%
\def\footnote{\protect\rmarkdownfootnote}

%%% Change title format to be more compact
\usepackage{titling}

% Create subtitle command for use in maketitle
\providecommand{\subtitle}[1]{
  \posttitle{
    \begin{center}\large#1\end{center}
    }
}

\setlength{\droptitle}{-2em}

  \title{UtilityFunctions}
    \pretitle{\vspace{\droptitle}\centering\huge}
  \posttitle{\par}
    \author{Sarah I. Murphy}
    \preauthor{\centering\large\emph}
  \postauthor{\par}
      \predate{\centering\large\emph}
  \postdate{\par}
    \date{2/1/2020}


\begin{document}
\maketitle

\hypertarget{utility-functions}{%
\subsection{Utility Functions}\label{utility-functions}}

\begin{itemize}
\tightlist
\item
  muAtNewTemp

  \begin{itemize}
  \tightlist
  \item
    Purpose: Calculate the new mu parameter at new temperature.
  \item
    Params:

    \begin{itemize}
    \tightlist
    \item
      newTemp: the new temperature for which we calculate mu
    \item
      oldMu: the previous mu value to adjust
    \item
      oldTemp: the temperature corresponding to previous mu
    \item
      T0: parameter used to calculate new mu
    \item
      This uses the Ratkowsky square model which describes the effect of
      temperature on the growth of microorganisms
    \item
      The paper it comes from:
      \url{https://www.ncbi.nlm.nih.gov/pubmed/22417595}
    \item
      Sam checked equation and it is correct
    \item
      The T0 is estimate dto be -3.62C based on growth curves of
      Paenibacillus ordorifer obtained at 4, 7, and 32C in BHI broth
      (N.H. Martin unpublished data)
    \end{itemize}
  \end{itemize}
\end{itemize}

\begin{Shaded}
\begin{Highlighting}[]
\NormalTok{muAtNewTemp <-}\StringTok{ }\ControlFlowTok{function}\NormalTok{(newTemp, oldMu, }\DataTypeTok{oldTemp =} \DecValTok{6}\NormalTok{, }\DataTypeTok{T0 =} \FloatTok{-3.62}\NormalTok{) \{}
\NormalTok{  numerator <-}\StringTok{ }\NormalTok{newTemp }\OperatorTok{-}\StringTok{ }\NormalTok{T0}
\NormalTok{  denom <-}\StringTok{ }\NormalTok{oldTemp }\OperatorTok{-}\StringTok{ }\NormalTok{T0}
\NormalTok{  newMu <-}\StringTok{ }\NormalTok{((numerator }\OperatorTok{/}\StringTok{ }\NormalTok{denom)}\OperatorTok{^}\DecValTok{2}\NormalTok{) }\OperatorTok{*}\StringTok{ }\NormalTok{oldMu}
  
  \KeywordTok{return}\NormalTok{(newMu)}
\NormalTok{\}}
\end{Highlighting}
\end{Shaded}

\begin{itemize}
\tightlist
\item
  adjustLag

  \begin{itemize}
  \tightlist
  \item
    Purpose: Adjust the lag phase based on the Zwietering 1994 paper.
  \item
    Params:

    \begin{itemize}
    \tightlist
    \item
      t: the current timestep
    \item
      oldLag: the lag time at the previous temperature
    \item
      newLag: the lag time at the current temparature.
    \item
      restartExp: If true then lag phase restarts even if already in
      exponential growth phase.
    \item
      adjustmentConstant: The amount to adjust lag, the paper recommends
      0.25
    \end{itemize}
  \end{itemize}
\end{itemize}

\begin{Shaded}
\begin{Highlighting}[]
\NormalTok{adjustLag <-}\StringTok{ }\ControlFlowTok{function}\NormalTok{ (t, oldLag, newLag, }\DataTypeTok{restartExp =}\NormalTok{ T, }\DataTypeTok{adjustmentConstant =} \FloatTok{0.25}\NormalTok{) \{}
  \CommentTok{#determine the amount of lag phase completed}
\NormalTok{  remainingLag <-}\StringTok{ }\DecValTok{1} \OperatorTok{-}\StringTok{ }\NormalTok{(t }\OperatorTok{/}\StringTok{ }\NormalTok{oldLag)}
  \ControlFlowTok{if}\NormalTok{(restartExp) \{}
\NormalTok{  remainingLag <-}\StringTok{ }\KeywordTok{ifelse}\NormalTok{(remainingLag }\OperatorTok{<}\StringTok{ }\DecValTok{0}\NormalTok{, }\DecValTok{0}\NormalTok{, remainingLag)}
\NormalTok{  \}}
  \ControlFlowTok{else}\NormalTok{ \{}
\NormalTok{  adjustedLag <-}\StringTok{ }\KeywordTok{ifelse}\NormalTok{(remainingLag }\OperatorTok{<=}\DecValTok{0}\NormalTok{, oldLag,}
\NormalTok{                        t }\OperatorTok{+}\StringTok{ }\NormalTok{remainingLag }\OperatorTok{*}\StringTok{ }\NormalTok{newLag }\OperatorTok{+}\StringTok{ }\NormalTok{adjustmentConstant}\OperatorTok{*}\NormalTok{newLag)}
\NormalTok{  \}}
  
\NormalTok{  adjustedLag <-}\StringTok{ }\NormalTok{t }\OperatorTok{+}\StringTok{ }\NormalTok{remainingLag}\OperatorTok{*}\NormalTok{newLag }\OperatorTok{+}\StringTok{ }\NormalTok{adjustmentConstant}\OperatorTok{*}\NormalTok{newLag}
  \KeywordTok{return}\NormalTok{(adjustedLag)}
\NormalTok{\}}
\end{Highlighting}
\end{Shaded}

\begin{itemize}
\tightlist
\item
  lagAtNewTemp

  \begin{itemize}
  \tightlist
  \item
    Purpose: Calculate the new lag parameter at new temperature.
  \item
    Params:

    \begin{itemize}
    \tightlist
    \item
      newTemp: the new temperature for which we calculate lag
    \item
      oldLag: the previous lag value to adjust
    \item
      oldTemp: the temperature corresponding to previous lag
    \item
      T0: parameter used to calculate new lag
    \end{itemize}
  \end{itemize}
\end{itemize}

\begin{Shaded}
\begin{Highlighting}[]
\NormalTok{lagAtNewTemp <-}\StringTok{ }\ControlFlowTok{function}\NormalTok{ (t, newTemp, oldLag, }\DataTypeTok{oldTemp =} \DecValTok{6}\NormalTok{, }\DataTypeTok{T0 =} \FloatTok{-3.62}\NormalTok{) \{}
\NormalTok{  numerator <-}\StringTok{ }\NormalTok{oldTemp }\OperatorTok{-}\NormalTok{T0}
\NormalTok{  denom <-}\StringTok{ }\NormalTok{newTemp }\OperatorTok{-}\StringTok{ }\NormalTok{T0}
\NormalTok{  newLag <-}\StringTok{ }\NormalTok{( (numerator }\OperatorTok{/}\StringTok{ }\NormalTok{denom)}\OperatorTok{^}\DecValTok{2}\NormalTok{) }\OperatorTok{*}\StringTok{ }\NormalTok{oldLag}
  \KeywordTok{return}\NormalTok{(newLag)}
\NormalTok{\}}

\NormalTok{getPrevRow <-}\StringTok{ }\ControlFlowTok{function}\NormalTok{(df, sim_run, half_gallon, day) \{}
\NormalTok{  old_temp <-}\StringTok{ }\NormalTok{df[df}\OperatorTok{$}\NormalTok{BT }\OperatorTok{==}\StringTok{ }\NormalTok{sim_run }\OperatorTok{&}\StringTok{ }\NormalTok{df}\OperatorTok{$}\NormalTok{half_gal }\OperatorTok{==}\StringTok{ }\NormalTok{half_gallon }\OperatorTok{&}\StringTok{ }\NormalTok{df}\OperatorTok{$}\NormalTok{day}\OperatorTok{==}\NormalTok{day}\DecValTok{-1}\NormalTok{,] }
\NormalTok{\}}
\end{Highlighting}
\end{Shaded}

\textless{}\textless{}\textless{}\textless{}\textless{}\textless{}\textless{}
HEAD + buchanan\_log10N + Purpose: + Params: ======= + Growth Models +
All equations are copied from the Nlms package in R + URL:
\url{https://rdrr.io/cran/nlsMicrobio/src/R/growthmodels.R}

\begin{quote}
\begin{quote}
\begin{quote}
\begin{quote}
\begin{quote}
\begin{quote}
\begin{quote}
7c72935bc55f0c36b67a5d3665940ed245d85cc1
\end{quote}
\end{quote}
\end{quote}
\end{quote}
\end{quote}
\end{quote}
\end{quote}

\begin{Shaded}
\begin{Highlighting}[]
\NormalTok{buchanan_log10N =}\StringTok{ }\ControlFlowTok{function}\NormalTok{(t,lag,mumax,LOG10N0,LOG10Nmax)\{}
\NormalTok{  ans <-}\StringTok{ }\NormalTok{LOG10N0 }\OperatorTok{+}\StringTok{ }\NormalTok{(t }\OperatorTok{>=}\StringTok{ }\NormalTok{lag) }\OperatorTok{*}\StringTok{ }\NormalTok{(t }\OperatorTok{<=}\StringTok{ }\NormalTok{(lag }\OperatorTok{+}\StringTok{ }\NormalTok{(LOG10Nmax }\OperatorTok{-}\StringTok{ }\NormalTok{LOG10N0) }\OperatorTok{*}\StringTok{ }\KeywordTok{log}\NormalTok{(}\DecValTok{10}\NormalTok{)}\OperatorTok{/}\NormalTok{mumax)) }\OperatorTok{*}\StringTok{ }\NormalTok{mumax }\OperatorTok{*}\StringTok{ }\NormalTok{(t }\OperatorTok{-}\StringTok{ }\NormalTok{lag)}\OperatorTok{/}\KeywordTok{log}\NormalTok{(}\DecValTok{10}\NormalTok{) }\OperatorTok{+}\StringTok{ }\NormalTok{(t }\OperatorTok{>=}\StringTok{ }\NormalTok{lag) }\OperatorTok{*}\StringTok{ }\NormalTok{(t }\OperatorTok{>}\StringTok{ }\NormalTok{(lag }\OperatorTok{+}\StringTok{ }\NormalTok{(LOG10Nmax }\OperatorTok{-}\StringTok{ }\NormalTok{LOG10N0) }\OperatorTok{*}\StringTok{ }\KeywordTok{log}\NormalTok{(}\DecValTok{10}\NormalTok{)}\OperatorTok{/}\NormalTok{mumax)) }\OperatorTok{*}\StringTok{ }\NormalTok{(LOG10Nmax }\OperatorTok{-}\StringTok{     }\NormalTok{LOG10N0)}
  \KeywordTok{return}\NormalTok{(ans)}
\NormalTok{\}}
\end{Highlighting}
\end{Shaded}

\begin{itemize}
\tightlist
\item
  gompertz\_log10N

  \begin{itemize}
  \tightlist
  \item
    Purpose:
  \item
    Params:
  \end{itemize}
\end{itemize}

\begin{Shaded}
\begin{Highlighting}[]
\NormalTok{gompertz_log10N =}\StringTok{ }\ControlFlowTok{function}\NormalTok{(t,lag,mumax,LOG10N0,LOG10Nmax) \{}
\NormalTok{  ans <-}\StringTok{ }\NormalTok{LOG10N0 }\OperatorTok{+}\StringTok{ }\NormalTok{(LOG10Nmax }\OperatorTok{-}\StringTok{ }\NormalTok{LOG10N0) }\OperatorTok{*}\StringTok{ }\KeywordTok{exp}\NormalTok{(}\OperatorTok{-}\KeywordTok{exp}\NormalTok{(mumax }\OperatorTok{*}\StringTok{ }\KeywordTok{exp}\NormalTok{(}\DecValTok{1}\NormalTok{) }\OperatorTok{*}\StringTok{ }\NormalTok{(lag }\OperatorTok{-}\StringTok{ }\NormalTok{t)}\OperatorTok{/}\NormalTok{((LOG10Nmax }\OperatorTok{-}\StringTok{ }\NormalTok{LOG10N0) }\OperatorTok{*}\StringTok{ }\KeywordTok{log}\NormalTok{(}\DecValTok{10}\NormalTok{)) }\OperatorTok{+}\StringTok{ }\DecValTok{1}\NormalTok{))}
  \KeywordTok{return}\NormalTok{(ans)}
\NormalTok{\}}
\end{Highlighting}
\end{Shaded}

\begin{itemize}
\tightlist
\item
  baranyi\_log10N

  \begin{itemize}
  \tightlist
  \item
    Purpose:
  \item
    Params:
  \end{itemize}
\end{itemize}

\begin{Shaded}
\begin{Highlighting}[]
\NormalTok{baranyi_log10N =}\StringTok{ }\ControlFlowTok{function}\NormalTok{(t,lag,mumax,LOG10N0,LOG10Nmax) \{}
\NormalTok{  ans <-}\StringTok{ }\NormalTok{LOG10Nmax }\OperatorTok{+}\StringTok{ }\KeywordTok{log10}\NormalTok{((}\OperatorTok{-}\DecValTok{1} \OperatorTok{+}\StringTok{ }\KeywordTok{exp}\NormalTok{(mumax }\OperatorTok{*}\StringTok{ }\NormalTok{lag) }\OperatorTok{+}\StringTok{ }\KeywordTok{exp}\NormalTok{(mumax }\OperatorTok{*}\StringTok{ }\NormalTok{t))}\OperatorTok{/}\NormalTok{(}\KeywordTok{exp}\NormalTok{(mumax }\OperatorTok{*}\StringTok{ }\NormalTok{t) }\OperatorTok{-}\StringTok{ }\DecValTok{1} \OperatorTok{+}\StringTok{ }\KeywordTok{exp}\NormalTok{(mumax }\OperatorTok{*}\StringTok{ }\NormalTok{lag) }\OperatorTok{*}\StringTok{ }\DecValTok{10}\OperatorTok{^}\NormalTok{(LOG10Nmax }\OperatorTok{-}\StringTok{ }\NormalTok{LOG10N0)))}
  \KeywordTok{return}\NormalTok{(ans)}
\NormalTok{\}}
\end{Highlighting}
\end{Shaded}

\textless{}\textless{}\textless{}\textless{}\textless{}\textless{}\textless{}
HEAD + log10N wrapper function + Purpose: calculate log10N and
implements the growth model ======= + Function to calculate log10N;
wrapper function because it calls the proper model + Purpose: This
implements the growth model + You need to import a CSV file with the
best fitting model pre-selected + Your CSV file should have a column
that has a column name, model\_name and filled out with buchanan,
baranyi, or gompertz
\textgreater{}\textgreater{}\textgreater{}\textgreater{}\textgreater{}\textgreater{}\textgreater{}
7c72935bc55f0c36b67a5d3665940ed245d85cc1

\begin{Shaded}
\begin{Highlighting}[]
\NormalTok{log10N_func <-}\StringTok{ }\ControlFlowTok{function}\NormalTok{(t, lag, mumax, LOG10N0, LOG10Nmax, }\DataTypeTok{model_name=}\StringTok{"buchanan"}\NormalTok{) \{}
  \ControlFlowTok{if}\NormalTok{ (model_name }\OperatorTok{==}\StringTok{ "buchanan"}\NormalTok{) \{}
    \KeywordTok{return}\NormalTok{(}\KeywordTok{buchanan_log10N}\NormalTok{(t, lag, mumax, LOG10N0, LOG10Nmax) )}
\NormalTok{  \}}
  \ControlFlowTok{else} \ControlFlowTok{if}\NormalTok{(model_name }\OperatorTok{==}\StringTok{ 'baranyi'}\NormalTok{) \{}
    \KeywordTok{return}\NormalTok{(}\KeywordTok{baranyi_log10N}\NormalTok{(t, lag, mumax, LOG10N0, LOG10Nmax) )}
\NormalTok{  \}}
  \ControlFlowTok{else} \ControlFlowTok{if}\NormalTok{(model_name }\OperatorTok{==}\StringTok{ 'gompertz'}\NormalTok{) \{}
    \KeywordTok{return}\NormalTok{(}\KeywordTok{gompertz_log10N}\NormalTok{(t, lag, mumax, LOG10N0, LOG10Nmax) )}
\NormalTok{  \}}
  \ControlFlowTok{else}\NormalTok{ \{}
    \KeywordTok{stop}\NormalTok{(}\KeywordTok{paste0}\NormalTok{(model_name, }\StringTok{" is not a valid model name. Must be one of buchanan, baranyi, gompertz"}\NormalTok{))}
\NormalTok{  \}}
\NormalTok{\}}
\end{Highlighting}
\end{Shaded}


\end{document}
