\documentclass[]{article}
\usepackage{lmodern}
\usepackage{amssymb,amsmath}
\usepackage{ifxetex,ifluatex}
\usepackage{fixltx2e} % provides \textsubscript
\ifnum 0\ifxetex 1\fi\ifluatex 1\fi=0 % if pdftex
  \usepackage[T1]{fontenc}
  \usepackage[utf8]{inputenc}
\else % if luatex or xelatex
  \ifxetex
    \usepackage{mathspec}
  \else
    \usepackage{fontspec}
  \fi
  \defaultfontfeatures{Ligatures=TeX,Scale=MatchLowercase}
\fi
% use upquote if available, for straight quotes in verbatim environments
\IfFileExists{upquote.sty}{\usepackage{upquote}}{}
% use microtype if available
\IfFileExists{microtype.sty}{%
\usepackage{microtype}
\UseMicrotypeSet[protrusion]{basicmath} % disable protrusion for tt fonts
}{}
\usepackage[margin=1in]{geometry}
\usepackage{hyperref}
\hypersetup{unicode=true,
            pdftitle={MCsim},
            pdfauthor={Sarah I. Murphy, Samantha Lau, Timothy Lott, Aljosa Trmcic},
            pdfborder={0 0 0},
            breaklinks=true}
\urlstyle{same}  % don't use monospace font for urls
\usepackage{color}
\usepackage{fancyvrb}
\newcommand{\VerbBar}{|}
\newcommand{\VERB}{\Verb[commandchars=\\\{\}]}
\DefineVerbatimEnvironment{Highlighting}{Verbatim}{commandchars=\\\{\}}
% Add ',fontsize=\small' for more characters per line
\usepackage{framed}
\definecolor{shadecolor}{RGB}{248,248,248}
\newenvironment{Shaded}{\begin{snugshade}}{\end{snugshade}}
\newcommand{\AlertTok}[1]{\textcolor[rgb]{0.94,0.16,0.16}{#1}}
\newcommand{\AnnotationTok}[1]{\textcolor[rgb]{0.56,0.35,0.01}{\textbf{\textit{#1}}}}
\newcommand{\AttributeTok}[1]{\textcolor[rgb]{0.77,0.63,0.00}{#1}}
\newcommand{\BaseNTok}[1]{\textcolor[rgb]{0.00,0.00,0.81}{#1}}
\newcommand{\BuiltInTok}[1]{#1}
\newcommand{\CharTok}[1]{\textcolor[rgb]{0.31,0.60,0.02}{#1}}
\newcommand{\CommentTok}[1]{\textcolor[rgb]{0.56,0.35,0.01}{\textit{#1}}}
\newcommand{\CommentVarTok}[1]{\textcolor[rgb]{0.56,0.35,0.01}{\textbf{\textit{#1}}}}
\newcommand{\ConstantTok}[1]{\textcolor[rgb]{0.00,0.00,0.00}{#1}}
\newcommand{\ControlFlowTok}[1]{\textcolor[rgb]{0.13,0.29,0.53}{\textbf{#1}}}
\newcommand{\DataTypeTok}[1]{\textcolor[rgb]{0.13,0.29,0.53}{#1}}
\newcommand{\DecValTok}[1]{\textcolor[rgb]{0.00,0.00,0.81}{#1}}
\newcommand{\DocumentationTok}[1]{\textcolor[rgb]{0.56,0.35,0.01}{\textbf{\textit{#1}}}}
\newcommand{\ErrorTok}[1]{\textcolor[rgb]{0.64,0.00,0.00}{\textbf{#1}}}
\newcommand{\ExtensionTok}[1]{#1}
\newcommand{\FloatTok}[1]{\textcolor[rgb]{0.00,0.00,0.81}{#1}}
\newcommand{\FunctionTok}[1]{\textcolor[rgb]{0.00,0.00,0.00}{#1}}
\newcommand{\ImportTok}[1]{#1}
\newcommand{\InformationTok}[1]{\textcolor[rgb]{0.56,0.35,0.01}{\textbf{\textit{#1}}}}
\newcommand{\KeywordTok}[1]{\textcolor[rgb]{0.13,0.29,0.53}{\textbf{#1}}}
\newcommand{\NormalTok}[1]{#1}
\newcommand{\OperatorTok}[1]{\textcolor[rgb]{0.81,0.36,0.00}{\textbf{#1}}}
\newcommand{\OtherTok}[1]{\textcolor[rgb]{0.56,0.35,0.01}{#1}}
\newcommand{\PreprocessorTok}[1]{\textcolor[rgb]{0.56,0.35,0.01}{\textit{#1}}}
\newcommand{\RegionMarkerTok}[1]{#1}
\newcommand{\SpecialCharTok}[1]{\textcolor[rgb]{0.00,0.00,0.00}{#1}}
\newcommand{\SpecialStringTok}[1]{\textcolor[rgb]{0.31,0.60,0.02}{#1}}
\newcommand{\StringTok}[1]{\textcolor[rgb]{0.31,0.60,0.02}{#1}}
\newcommand{\VariableTok}[1]{\textcolor[rgb]{0.00,0.00,0.00}{#1}}
\newcommand{\VerbatimStringTok}[1]{\textcolor[rgb]{0.31,0.60,0.02}{#1}}
\newcommand{\WarningTok}[1]{\textcolor[rgb]{0.56,0.35,0.01}{\textbf{\textit{#1}}}}
\usepackage{graphicx,grffile}
\makeatletter
\def\maxwidth{\ifdim\Gin@nat@width>\linewidth\linewidth\else\Gin@nat@width\fi}
\def\maxheight{\ifdim\Gin@nat@height>\textheight\textheight\else\Gin@nat@height\fi}
\makeatother
% Scale images if necessary, so that they will not overflow the page
% margins by default, and it is still possible to overwrite the defaults
% using explicit options in \includegraphics[width, height, ...]{}
\setkeys{Gin}{width=\maxwidth,height=\maxheight,keepaspectratio}
\IfFileExists{parskip.sty}{%
\usepackage{parskip}
}{% else
\setlength{\parindent}{0pt}
\setlength{\parskip}{6pt plus 2pt minus 1pt}
}
\setlength{\emergencystretch}{3em}  % prevent overfull lines
\providecommand{\tightlist}{%
  \setlength{\itemsep}{0pt}\setlength{\parskip}{0pt}}
\setcounter{secnumdepth}{0}
% Redefines (sub)paragraphs to behave more like sections
\ifx\paragraph\undefined\else
\let\oldparagraph\paragraph
\renewcommand{\paragraph}[1]{\oldparagraph{#1}\mbox{}}
\fi
\ifx\subparagraph\undefined\else
\let\oldsubparagraph\subparagraph
\renewcommand{\subparagraph}[1]{\oldsubparagraph{#1}\mbox{}}
\fi

%%% Use protect on footnotes to avoid problems with footnotes in titles
\let\rmarkdownfootnote\footnote%
\def\footnote{\protect\rmarkdownfootnote}

%%% Change title format to be more compact
\usepackage{titling}

% Create subtitle command for use in maketitle
\providecommand{\subtitle}[1]{
  \posttitle{
    \begin{center}\large#1\end{center}
    }
}

\setlength{\droptitle}{-2em}

  \title{MCsim}
    \pretitle{\vspace{\droptitle}\centering\huge}
  \posttitle{\par}
    \author{Sarah I. Murphy, Samantha Lau, Timothy Lott, Aljosa Trmcic}
    \preauthor{\centering\large\emph}
  \postauthor{\par}
      \predate{\centering\large\emph}
  \postdate{\par}
    \date{1/28/2020}


\begin{document}
\maketitle

The authors listed above are current contributors to this file. Earlier
versions were the work of Ariel Buehler and Michael Phillips. Review
README.md
{[}\url{https://github.com/FSL-MQIP/MC-2020/blob/master/README.md}{]}
for more information.

\begin{Shaded}
\begin{Highlighting}[]
\KeywordTok{set.seed}\NormalTok{(}\DecValTok{1}\NormalTok{) }\CommentTok{#set seed for reproducibility}
\end{Highlighting}
\end{Shaded}

\hypertarget{load-packages-data}{%
\subsection{Load packages \& data}\label{load-packages-data}}

\begin{Shaded}
\begin{Highlighting}[]
\CommentTok{#load packages}
\KeywordTok{library}\NormalTok{(readr)}

\CommentTok{#Name the files to be read}
\NormalTok{frequency_file <-}\StringTok{ "Frequency.csv"}
\NormalTok{growth_file <-}\StringTok{ "GrowthParameters.csv"} \CommentTok{#make sure this contains growth parameters & growth model name}
\NormalTok{init_file <-}\StringTok{ "InitialCountsMPN.csv"}
\NormalTok{temp_stages_file <-}\StringTok{ "temp_stages.csv"}

\CommentTok{#Import frequency data and get the rpoB allelic type}
\NormalTok{freq_import <-}\StringTok{ }\KeywordTok{read.csv}\NormalTok{(frequency_file, }\DataTypeTok{stringsAsFactors =} \OtherTok{FALSE}\NormalTok{, }\DataTypeTok{header =} \OtherTok{TRUE}\NormalTok{)}
\NormalTok{freq_data =}\StringTok{ }\NormalTok{freq_import}\OperatorTok{$}\NormalTok{rpoB.allelic.type}

\CommentTok{#Import growth parameter data}
\NormalTok{growth_import <-}\KeywordTok{read.csv}\NormalTok{(growth_file, }\DataTypeTok{stringsAsFactors =} \OtherTok{FALSE}\NormalTok{)}

\CommentTok{#Import initial count logMPN data}
\NormalTok{initialcount_import <-}\StringTok{ }\KeywordTok{read.csv}\NormalTok{(init_file, }\DataTypeTok{stringsAsFactors =} \OtherTok{FALSE}\NormalTok{)}
\CommentTok{#MPN Column}
\NormalTok{initialcount_data =}\StringTok{ }\NormalTok{initialcount_import[,}\DecValTok{3}\NormalTok{]}
\CommentTok{#LOG MPN Column}
\NormalTok{initialcountlog_data =}\StringTok{ }\NormalTok{initialcount_import[,}\DecValTok{4}\NormalTok{]}

\CommentTok{#Import temperature data}
\NormalTok{stages <-}\StringTok{ }\KeywordTok{read.csv}\NormalTok{(temp_stages_file, }\DataTypeTok{stringsAsFactors =}\NormalTok{ F, }\DataTypeTok{comment.char =} \StringTok{"#"}\NormalTok{)}
\end{Highlighting}
\end{Shaded}

\begin{verbatim}
## Warning in read.table(file = file, header = header, sep = sep,
## quote = quote, : incomplete final line found by readTableHeader on
## 'temp_stages.csv'
\end{verbatim}

\hypertarget{utility-functions}{%
\subsection{Utility Functions}\label{utility-functions}}

\begin{itemize}
\tightlist
\item
  muAtNewTemp

  \begin{itemize}
  \tightlist
  \item
    Purpose: Calculate the new mu parameter at new temperature.
  \item
    Params:

    \begin{itemize}
    \tightlist
    \item
      newTemp: the new temperature for which we calculate mu
    \item
      oldMu: the previous mu value to adjust
    \item
      oldTemp: the temperature corresponding to previous mu
    \item
      T0: parameter used to calculate new mu
    \item
      This uses the Ratkowsky square model which describes the effect of
      temperature on the growth of microorganisms
    \item
      The paper it comes from:
      \url{https://www.ncbi.nlm.nih.gov/pubmed/22417595}
    \item
      Sam checked equation and it is correct
    \item
      The T0 is estimate dto be -3.62C based on growth curves of
      Paenibacillus ordorifer obtained at 4, 7, and 32C in BHI broth
      (N.H. Martin unpublished data)
    \end{itemize}
  \end{itemize}
\end{itemize}

\begin{Shaded}
\begin{Highlighting}[]
\NormalTok{muAtNewTemp <-}\StringTok{ }\ControlFlowTok{function}\NormalTok{(newTemp, oldMu, }\DataTypeTok{oldTemp =} \DecValTok{6}\NormalTok{, }\DataTypeTok{T0 =} \FloatTok{-3.62}\NormalTok{) \{}
\NormalTok{  numerator <-}\StringTok{ }\NormalTok{newTemp }\OperatorTok{-}\StringTok{ }\NormalTok{T0}
\NormalTok{  denom <-}\StringTok{ }\NormalTok{oldTemp }\OperatorTok{-}\StringTok{ }\NormalTok{T0}
\NormalTok{  newMu <-}\StringTok{ }\NormalTok{((numerator }\OperatorTok{/}\StringTok{ }\NormalTok{denom)}\OperatorTok{^}\DecValTok{2}\NormalTok{) }\OperatorTok{*}\StringTok{ }\NormalTok{oldMu}
  
  \KeywordTok{return}\NormalTok{(newMu)}
\NormalTok{\}}
\end{Highlighting}
\end{Shaded}

\begin{itemize}
\tightlist
\item
  adjustLag

  \begin{itemize}
  \tightlist
  \item
    Purpose: Adjust the lag phase based on the Zwietering 1994 paper.
  \item
    Params:

    \begin{itemize}
    \tightlist
    \item
      t: the current timestep
    \item
      oldLag: the lag time at the previous temperature
    \item
      newLag: the lag time at the current temparature.
    \item
      restartExp: If true then lag phase restarts even if already in
      exponential growth phase.
    \item
      adjustmentConstant: The amount to adjust lag, the paper recommends
      0.25
    \end{itemize}
  \end{itemize}
\end{itemize}

\begin{Shaded}
\begin{Highlighting}[]
\NormalTok{adjustLag <-}\StringTok{ }\ControlFlowTok{function}\NormalTok{ (t, oldLag, newLag, }\DataTypeTok{restartExp =}\NormalTok{ T, }\DataTypeTok{adjustmentConstant =} \FloatTok{0.25}\NormalTok{) \{}
  \CommentTok{#determine the amount of lag phase completed}
\NormalTok{  remainingLag <-}\StringTok{ }\DecValTok{1} \OperatorTok{-}\StringTok{ }\NormalTok{(t }\OperatorTok{/}\StringTok{ }\NormalTok{oldLag)}
  \ControlFlowTok{if}\NormalTok{(restartExp) \{}
\NormalTok{  remainingLag <-}\StringTok{ }\KeywordTok{ifelse}\NormalTok{(remainingLag }\OperatorTok{<}\StringTok{ }\DecValTok{0}\NormalTok{, }\DecValTok{0}\NormalTok{, remainingLag)}
\NormalTok{  \}}
  \ControlFlowTok{else}\NormalTok{ \{}
\NormalTok{  adjustedLag <-}\StringTok{ }\KeywordTok{ifelse}\NormalTok{(remainingLag }\OperatorTok{<=}\DecValTok{0}\NormalTok{, oldLag,}
\NormalTok{                        t }\OperatorTok{+}\StringTok{ }\NormalTok{remainingLag }\OperatorTok{*}\StringTok{ }\NormalTok{newLag }\OperatorTok{+}\StringTok{ }\NormalTok{adjustmentConstant}\OperatorTok{*}\NormalTok{newLag)}
\NormalTok{  \}}
  
\NormalTok{  adjustedLag <-}\StringTok{ }\NormalTok{t }\OperatorTok{+}\StringTok{ }\NormalTok{remainingLag}\OperatorTok{*}\NormalTok{newLag }\OperatorTok{+}\StringTok{ }\NormalTok{adjustmentConstant}\OperatorTok{*}\NormalTok{newLag}
  \KeywordTok{return}\NormalTok{(adjustedLag)}
\NormalTok{\}}
\end{Highlighting}
\end{Shaded}

\begin{itemize}
\tightlist
\item
  lagAtNewTemp

  \begin{itemize}
  \tightlist
  \item
    Purpose: Calculate the new lag parameter at new temperature.
  \item
    Params:

    \begin{itemize}
    \tightlist
    \item
      newTemp: the new temperature for which we calculate lag
    \item
      oldLag: the previous lag value to adjust
    \item
      oldTemp: the temperature corresponding to previous lag
    \item
      T0: parameter used to calculate new lag
    \end{itemize}
  \end{itemize}
\end{itemize}

\begin{Shaded}
\begin{Highlighting}[]
\NormalTok{lagAtNewTemp <-}\StringTok{ }\ControlFlowTok{function}\NormalTok{ (t, newTemp, oldLag, }\DataTypeTok{oldTemp =} \DecValTok{6}\NormalTok{, }\DataTypeTok{T0 =} \FloatTok{-3.62}\NormalTok{) \{}
\NormalTok{  numerator <-}\StringTok{ }\NormalTok{oldTemp }\OperatorTok{-}\NormalTok{T0}
\NormalTok{  denom <-}\StringTok{ }\NormalTok{newTemp }\OperatorTok{-}\StringTok{ }\NormalTok{T0}
\NormalTok{  newLag <-}\StringTok{ }\NormalTok{( (numerator }\OperatorTok{/}\StringTok{ }\NormalTok{denom)}\OperatorTok{^}\DecValTok{2}\NormalTok{) }\OperatorTok{*}\StringTok{ }\NormalTok{oldLag}
  \KeywordTok{return}\NormalTok{(newLag)}
\NormalTok{\}}

\NormalTok{getPrevRow <-}\StringTok{ }\ControlFlowTok{function}\NormalTok{(df, sim_run, half_gallon, day) \{}
\NormalTok{  old_temp <-}\StringTok{ }\NormalTok{df[df}\OperatorTok{$}\NormalTok{BT }\OperatorTok{==}\StringTok{ }\NormalTok{sim_run }\OperatorTok{&}\StringTok{ }\NormalTok{df}\OperatorTok{$}\NormalTok{half_gal }\OperatorTok{==}\StringTok{ }\NormalTok{half_gallon }\OperatorTok{&}\StringTok{ }\NormalTok{df}\OperatorTok{$}\NormalTok{day}\OperatorTok{==}\NormalTok{day}\DecValTok{-1}\NormalTok{,] }
\NormalTok{\}}
\end{Highlighting}
\end{Shaded}

\textless{}\textless{}\textless{}\textless{}\textless{}\textless{}\textless{}
HEAD + buchanan\_log10N + Purpose: + Params: ======= + Growth Models +
All equations are copied from the Nlms package in R + URL:
\url{https://rdrr.io/cran/nlsMicrobio/src/R/growthmodels.R}

\begin{quote}
\begin{quote}
\begin{quote}
\begin{quote}
\begin{quote}
\begin{quote}
\begin{quote}
7c72935bc55f0c36b67a5d3665940ed245d85cc1
\end{quote}
\end{quote}
\end{quote}
\end{quote}
\end{quote}
\end{quote}
\end{quote}

\begin{Shaded}
\begin{Highlighting}[]
\NormalTok{buchanan_log10N =}\StringTok{ }\ControlFlowTok{function}\NormalTok{(t,lag,mumax,LOG10N0,LOG10Nmax)\{}
\NormalTok{  ans <-}\StringTok{ }\NormalTok{LOG10N0 }\OperatorTok{+}\StringTok{ }\NormalTok{(t }\OperatorTok{>=}\StringTok{ }\NormalTok{lag) }\OperatorTok{*}\StringTok{ }\NormalTok{(t }\OperatorTok{<=}\StringTok{ }\NormalTok{(lag }\OperatorTok{+}\StringTok{ }\NormalTok{(LOG10Nmax }\OperatorTok{-}\StringTok{ }\NormalTok{LOG10N0) }\OperatorTok{*}\StringTok{ }\KeywordTok{log}\NormalTok{(}\DecValTok{10}\NormalTok{)}\OperatorTok{/}\NormalTok{mumax)) }\OperatorTok{*}\StringTok{ }\NormalTok{mumax }\OperatorTok{*}\StringTok{ }\NormalTok{(t }\OperatorTok{-}\StringTok{ }\NormalTok{lag)}\OperatorTok{/}\KeywordTok{log}\NormalTok{(}\DecValTok{10}\NormalTok{) }\OperatorTok{+}\StringTok{ }\NormalTok{(t }\OperatorTok{>=}\StringTok{ }\NormalTok{lag) }\OperatorTok{*}\StringTok{ }\NormalTok{(t }\OperatorTok{>}\StringTok{ }\NormalTok{(lag }\OperatorTok{+}\StringTok{ }\NormalTok{(LOG10Nmax }\OperatorTok{-}\StringTok{ }\NormalTok{LOG10N0) }\OperatorTok{*}\StringTok{ }\KeywordTok{log}\NormalTok{(}\DecValTok{10}\NormalTok{)}\OperatorTok{/}\NormalTok{mumax)) }\OperatorTok{*}\StringTok{ }\NormalTok{(LOG10Nmax }\OperatorTok{-}\StringTok{     }\NormalTok{LOG10N0)}
  \KeywordTok{return}\NormalTok{(ans)}
\NormalTok{\}}
\end{Highlighting}
\end{Shaded}

\begin{itemize}
\tightlist
\item
  gompertz\_log10N

  \begin{itemize}
  \tightlist
  \item
    Purpose:
  \item
    Params:
  \end{itemize}
\end{itemize}

\begin{Shaded}
\begin{Highlighting}[]
\NormalTok{gompertz_log10N =}\StringTok{ }\ControlFlowTok{function}\NormalTok{(t,lag,mumax,LOG10N0,LOG10Nmax) \{}
\NormalTok{  ans <-}\StringTok{ }\NormalTok{LOG10N0 }\OperatorTok{+}\StringTok{ }\NormalTok{(LOG10Nmax }\OperatorTok{-}\StringTok{ }\NormalTok{LOG10N0) }\OperatorTok{*}\StringTok{ }\KeywordTok{exp}\NormalTok{(}\OperatorTok{-}\KeywordTok{exp}\NormalTok{(mumax }\OperatorTok{*}\StringTok{ }\KeywordTok{exp}\NormalTok{(}\DecValTok{1}\NormalTok{) }\OperatorTok{*}\StringTok{ }\NormalTok{(lag }\OperatorTok{-}\StringTok{ }\NormalTok{t)}\OperatorTok{/}\NormalTok{((LOG10Nmax }\OperatorTok{-}\StringTok{ }\NormalTok{LOG10N0) }\OperatorTok{*}\StringTok{ }\KeywordTok{log}\NormalTok{(}\DecValTok{10}\NormalTok{)) }\OperatorTok{+}\StringTok{ }\DecValTok{1}\NormalTok{))}
  \KeywordTok{return}\NormalTok{(ans)}
\NormalTok{\}}
\end{Highlighting}
\end{Shaded}

\begin{itemize}
\tightlist
\item
  baranyi\_log10N

  \begin{itemize}
  \tightlist
  \item
    Purpose:
  \item
    Params:
  \end{itemize}
\end{itemize}

\begin{Shaded}
\begin{Highlighting}[]
\NormalTok{baranyi_log10N =}\StringTok{ }\ControlFlowTok{function}\NormalTok{(t,lag,mumax,LOG10N0,LOG10Nmax) \{}
\NormalTok{  ans <-}\StringTok{ }\NormalTok{LOG10Nmax }\OperatorTok{+}\StringTok{ }\KeywordTok{log10}\NormalTok{((}\OperatorTok{-}\DecValTok{1} \OperatorTok{+}\StringTok{ }\KeywordTok{exp}\NormalTok{(mumax }\OperatorTok{*}\StringTok{ }\NormalTok{lag) }\OperatorTok{+}\StringTok{ }\KeywordTok{exp}\NormalTok{(mumax }\OperatorTok{*}\StringTok{ }\NormalTok{t))}\OperatorTok{/}\NormalTok{(}\KeywordTok{exp}\NormalTok{(mumax }\OperatorTok{*}\StringTok{ }\NormalTok{t) }\OperatorTok{-}\StringTok{ }\DecValTok{1} \OperatorTok{+}\StringTok{ }\KeywordTok{exp}\NormalTok{(mumax }\OperatorTok{*}\StringTok{ }\NormalTok{lag) }\OperatorTok{*}\StringTok{ }\DecValTok{10}\OperatorTok{^}\NormalTok{(LOG10Nmax }\OperatorTok{-}\StringTok{ }\NormalTok{LOG10N0)))}
  \KeywordTok{return}\NormalTok{(ans)}
\NormalTok{\}}
\end{Highlighting}
\end{Shaded}

\textless{}\textless{}\textless{}\textless{}\textless{}\textless{}\textless{}
HEAD + log10N wrapper function + Purpose: calculate log10N and
implements the growth model ======= + Function to calculate log10N;
wrapper function because it calls the proper model + Purpose: This
implements the growth model + You need to import a CSV file with the
best fitting model pre-selected + Your CSV file should have a column
that has a column name, model\_name and filled out with buchanan,
baranyi, or gompertz
\textgreater{}\textgreater{}\textgreater{}\textgreater{}\textgreater{}\textgreater{}\textgreater{}
7c72935bc55f0c36b67a5d3665940ed245d85cc1

\begin{Shaded}
\begin{Highlighting}[]
\NormalTok{log10N_func <-}\StringTok{ }\ControlFlowTok{function}\NormalTok{(t, lag, mumax, LOG10N0, LOG10Nmax, }\DataTypeTok{model_name=}\StringTok{"buchanan"}\NormalTok{) \{}
  \ControlFlowTok{if}\NormalTok{ (model_name }\OperatorTok{==}\StringTok{ "buchanan"}\NormalTok{) \{}
    \KeywordTok{return}\NormalTok{(}\KeywordTok{buchanan_log10N}\NormalTok{(t, lag, mumax, LOG10N0, LOG10Nmax) )}
\NormalTok{  \}}
  \ControlFlowTok{else} \ControlFlowTok{if}\NormalTok{(model_name }\OperatorTok{==}\StringTok{ 'baranyi'}\NormalTok{) \{}
    \KeywordTok{return}\NormalTok{(}\KeywordTok{baranyi_log10N}\NormalTok{(t, lag, mumax, LOG10N0, LOG10Nmax) )}
\NormalTok{  \}}
  \ControlFlowTok{else} \ControlFlowTok{if}\NormalTok{(model_name }\OperatorTok{==}\StringTok{ 'gompertz'}\NormalTok{) \{}
    \KeywordTok{return}\NormalTok{(}\KeywordTok{gompertz_log10N}\NormalTok{(t, lag, mumax, LOG10N0, LOG10Nmax) )}
\NormalTok{  \}}
  \ControlFlowTok{else}\NormalTok{ \{}
    \KeywordTok{stop}\NormalTok{(}\KeywordTok{paste0}\NormalTok{(model_name, }\StringTok{" is not a valid model name. Must be one of buchanan, baranyi, gompertz"}\NormalTok{))}
\NormalTok{  \}}
\NormalTok{\}}
\end{Highlighting}
\end{Shaded}

\hypertarget{data-frame-creation-and-setup}{%
\subsection{Data frame creation and
setup}\label{data-frame-creation-and-setup}}

\begin{itemize}
\tightlist
\item
  Set up data frame to store count at each day
\item
  SAM QUESTION: Why is day from start day to start day+nday-1, and not
  just from start day to nday
\end{itemize}

\begin{Shaded}
\begin{Highlighting}[]
\CommentTok{#Size is for n_sim bulk tanks, n_half_gal half gallon lots, n_day days}
\NormalTok{n_sim <-}\DecValTok{100}    \CommentTok{#1000 is for testing and exploring, experiments require at least 10k}
\NormalTok{n_halfgal <-}\DecValTok{10}
\NormalTok{n_day <-}\StringTok{ }\DecValTok{24}
\NormalTok{start_day <-}\StringTok{ }\DecValTok{1}

\CommentTok{#Repeat each element of the sequence 1..n_sim.Bulk tank data (MC runs)}
\NormalTok{BT <-}\StringTok{ }\KeywordTok{rep}\NormalTok{(}\KeywordTok{seq}\NormalTok{(}\DecValTok{1}\NormalTok{, n_sim), }\DataTypeTok{each =}\NormalTok{ n_halfgal }\OperatorTok{*}\StringTok{ }\NormalTok{n_day)}
\CommentTok{#Repeat the whole sequences times # of times}
\NormalTok{half_gal <-}\StringTok{ }\KeywordTok{rep}\NormalTok{(}\KeywordTok{seq}\NormalTok{(}\DecValTok{1}\NormalTok{, n_halfgal), }\DataTypeTok{times =}\NormalTok{ n_day }\OperatorTok{*}\StringTok{ }\NormalTok{n_sim)}
\CommentTok{#Vector of FALSE}
\NormalTok{AT <-}\StringTok{ }\KeywordTok{vector}\NormalTok{(}\DataTypeTok{mode=}\StringTok{"logical"}\NormalTok{, n_sim }\OperatorTok{*}\StringTok{ }\NormalTok{n_halfgal }\OperatorTok{*}\StringTok{ }\NormalTok{n_day)}
\CommentTok{#Repeat the days for each simulation run}
\NormalTok{day <-}\StringTok{ }\KeywordTok{rep}\NormalTok{(}\KeywordTok{rep}\NormalTok{(}\KeywordTok{seq}\NormalTok{(start_day, start_day}\OperatorTok{+}\NormalTok{n_day}\DecValTok{-1}\NormalTok{), }\DataTypeTok{each =}\NormalTok{ n_halfgal), }\DataTypeTok{times =}\NormalTok{ n_sim)}
\NormalTok{count <-}\StringTok{ }\KeywordTok{vector}\NormalTok{(}\DataTypeTok{mode =} \StringTok{"logical"}\NormalTok{, n_sim }\OperatorTok{*}\StringTok{ }\NormalTok{n_halfgal }\OperatorTok{*}\StringTok{ }\NormalTok{n_day)}

\CommentTok{#matrix with columns:}
\CommentTok{#  BT   half_gal    AT    day   count}
\NormalTok{data <-}\StringTok{ }\KeywordTok{data.frame}\NormalTok{(BT, half_gal, AT, day, count)}
\end{Highlighting}
\end{Shaded}

\hypertarget{calculate-samples-used-in-the-monte-carlo}{%
\subsection{Calculate samples used in the monte
carlo}\label{calculate-samples-used-in-the-monte-carlo}}

\begin{itemize}
\tightlist
\item
  Now sample the MPN distributions and the temperature distribution
\end{itemize}

\begin{Shaded}
\begin{Highlighting}[]
\CommentTok{#Sample logMPN from normal distribution}
\NormalTok{logMPN_mean <-}\StringTok{ }\KeywordTok{c}\NormalTok{(}\OperatorTok{-}\FloatTok{0.7226627}\NormalTok{)}
\NormalTok{logMPN_sd <-}\StringTok{ }\KeywordTok{c}\NormalTok{(.}\DecValTok{9901429}\NormalTok{)}
\NormalTok{logMPN_samp =}\StringTok{ }\KeywordTok{rnorm}\NormalTok{(n_sim, logMPN_mean, logMPN_sd)}
\NormalTok{MPN_samp =}\StringTok{ }\DecValTok{10}\OperatorTok{^}\NormalTok{logMPN_samp}
\NormalTok{MPN_samp_halfgal =}\StringTok{ }\NormalTok{MPN_samp }\OperatorTok{*}\StringTok{ }\DecValTok{1900} \CommentTok{#MPN per half gallon (1892.71 mL in half gallon)}

\CommentTok{#Generate initial MPN for each half gallon from Poisson distribution}
\CommentTok{#Also sample AT for each half gallon}
\NormalTok{MPN_init<-}\KeywordTok{vector}\NormalTok{()}
\NormalTok{allele <-}\StringTok{ }\KeywordTok{vector}\NormalTok{()}
\NormalTok{temps <-}\StringTok{ }\KeywordTok{vector}\NormalTok{()}
\ControlFlowTok{for}\NormalTok{ (i }\ControlFlowTok{in} \DecValTok{1}\OperatorTok{:}\NormalTok{n_sim)\{}
\NormalTok{  MPN_init_samp <-}\KeywordTok{rep}\NormalTok{(}\KeywordTok{rpois}\NormalTok{(n_halfgal, MPN_samp_halfgal[i]), }\DataTypeTok{times =}\NormalTok{ n_day)}
\NormalTok{  MPN_init<-}\KeywordTok{c}\NormalTok{(MPN_init, MPN_init_samp)}
\NormalTok{  allele_samp <-}\StringTok{ }\KeywordTok{rep}\NormalTok{(}\KeywordTok{sample}\NormalTok{(freq_data, n_halfgal, }\DataTypeTok{replace =}\NormalTok{ T), }\DataTypeTok{times =}\NormalTok{ n_day)}
\NormalTok{  allele <-}\StringTok{ }\KeywordTok{c}\NormalTok{(allele, allele_samp)}
  \CommentTok{#now calculate temp}
  \ControlFlowTok{for}\NormalTok{ (j }\ControlFlowTok{in} \DecValTok{1}\OperatorTok{:}\KeywordTok{nrow}\NormalTok{(stages))\{}
\NormalTok{    stage_row <-}\StringTok{ }\NormalTok{stages[j, ]}
\NormalTok{    n_times <-}\StringTok{ }\NormalTok{stage_row}\OperatorTok{$}\NormalTok{endTime }\OperatorTok{-}\StringTok{ }\NormalTok{stage_row}\OperatorTok{$}\NormalTok{beginTime }\OperatorTok{+}\StringTok{ }\DecValTok{1}
\NormalTok{    params <-}\StringTok{ }\KeywordTok{as.numeric}\NormalTok{(}\KeywordTok{unlist}\NormalTok{(}\KeywordTok{strsplit}\NormalTok{(stage_row}\OperatorTok{$}\NormalTok{parameters, }\StringTok{" "}\NormalTok{)))}
\NormalTok{    temp_mean <-}\StringTok{ }\NormalTok{params[[}\DecValTok{1}\NormalTok{]]}
\NormalTok{    temp_sd <-}\StringTok{ }\NormalTok{params[[}\DecValTok{2}\NormalTok{]]}
\NormalTok{    temp_sample <-}\StringTok{ }\KeywordTok{rep}\NormalTok{(}\KeywordTok{rnorm}\NormalTok{(n_halfgal, temp_mean, temp_sd), }\DataTypeTok{times =}\NormalTok{ n_times)}
\NormalTok{    temps <-}\StringTok{ }\KeywordTok{c}\NormalTok{(temps, temp_sample)}
\NormalTok{  \}}
\NormalTok{\}}
\CommentTok{#add in temperature}
\NormalTok{data}\OperatorTok{$}\NormalTok{temp <-}\StringTok{ }\NormalTok{temps}

\CommentTok{#Convert MPN_init from half-gallon to mLs}
\NormalTok{MPN_init_mL <-}\StringTok{ }\NormalTok{MPN_init }\OperatorTok{/}\StringTok{ }\DecValTok{1900}
\CommentTok{#remove 0's from the data and replace with detection limit}
\NormalTok{MPN_init_mL[MPN_init_mL }\OperatorTok{==}\StringTok{ }\DecValTok{0}\NormalTok{] <-}\StringTok{ }\FloatTok{0.01}\NormalTok{;}

\CommentTok{#Now we add in those calculations to our original dataframe}
\NormalTok{data}\OperatorTok{$}\NormalTok{logMPN_init <-}\StringTok{ }\KeywordTok{log10}\NormalTok{(MPN_init_mL) }\CommentTok{#Add initial logMPN to data frame}
\NormalTok{data}\OperatorTok{$}\NormalTok{AT<-allele }\CommentTok{#Add in AT data}

\NormalTok{data}\OperatorTok{$}\NormalTok{newTemp <-}\StringTok{ }\KeywordTok{vector}\NormalTok{(}\DataTypeTok{mode=}\StringTok{"logical"}\NormalTok{, n_sim }\OperatorTok{*}\StringTok{ }\NormalTok{n_halfgal }\OperatorTok{*}\StringTok{ }\NormalTok{n_day)}
\NormalTok{data}\OperatorTok{$}\NormalTok{newMu<-}\StringTok{ }\KeywordTok{vector}\NormalTok{(}\DataTypeTok{mode=}\StringTok{"logical"}\NormalTok{, n_sim }\OperatorTok{*}\StringTok{ }\NormalTok{n_halfgal }\OperatorTok{*}\StringTok{ }\NormalTok{n_day)}
\NormalTok{data}\OperatorTok{$}\NormalTok{newLag <-}\StringTok{ }\KeywordTok{vector}\NormalTok{(}\DataTypeTok{mode=}\StringTok{"logical"}\NormalTok{, n_sim }\OperatorTok{*}\StringTok{ }\NormalTok{n_halfgal }\OperatorTok{*}\StringTok{ }\NormalTok{n_day)}
\end{Highlighting}
\end{Shaded}

\begin{Shaded}
\begin{Highlighting}[]
\CommentTok{##Now we will calculate the log10N for each row in the data frame}
\CommentTok{##Get the AT and day from the data frame, get growth parameters depending on the AT}
\CommentTok{# Simulation   ----}
\ControlFlowTok{for}\NormalTok{ (i }\ControlFlowTok{in} \DecValTok{1}\OperatorTok{:}\NormalTok{(n_sim }\OperatorTok{*}\NormalTok{n_halfgal }\OperatorTok{*}\StringTok{ }\NormalTok{n_day))\{}
  \CommentTok{#Find row in growth parameter data that corresponds to allele sample}
\NormalTok{  allele_index <-}\StringTok{ }\KeywordTok{which}\NormalTok{(growth_import}\OperatorTok{$}\NormalTok{rpoBAT }\OperatorTok{==}\StringTok{ }\NormalTok{data}\OperatorTok{$}\NormalTok{AT[i]) }
\NormalTok{  row <-}\StringTok{ }\NormalTok{data[i, ]}
\NormalTok{  prev_row <-}\StringTok{ }\KeywordTok{getPrevRow}\NormalTok{(data, row}\OperatorTok{$}\NormalTok{BT, row}\OperatorTok{$}\NormalTok{half_gal, row}\OperatorTok{$}\NormalTok{day)}
\NormalTok{  update <-}\StringTok{ }\KeywordTok{ifelse}\NormalTok{(}\KeywordTok{nrow}\NormalTok{(prev_row) }\OperatorTok{==}\StringTok{ }\DecValTok{0} \OperatorTok{||}\StringTok{ }\NormalTok{row}\OperatorTok{$}\NormalTok{temp }\OperatorTok{!=}\StringTok{ }\NormalTok{prev_row}\OperatorTok{$}\NormalTok{temp, T, F)}
  \CommentTok{#calculate the new growth parameters using the square root model and our}
  \CommentTok{#sampled temperature}
  
\NormalTok{    newT <-}\StringTok{ }\NormalTok{row}\OperatorTok{$}\NormalTok{temp}
\NormalTok{    newLag <-}\StringTok{ }\KeywordTok{ifelse}\NormalTok{(update, }
                     \KeywordTok{lagAtNewTemp}\NormalTok{(row}\OperatorTok{$}\NormalTok{day, newT, growth_import}\OperatorTok{$}\NormalTok{lag[allele_index]),}
\NormalTok{                     prev_row}\OperatorTok{$}\NormalTok{newLag)}
\NormalTok{     old_lag <-}\StringTok{ }\KeywordTok{ifelse}\NormalTok{(row}\OperatorTok{$}\NormalTok{day}\OperatorTok{==}\DecValTok{1}\NormalTok{, newLag, prev_row}\OperatorTok{$}\NormalTok{newLag )}
     \CommentTok{# newLag <- ifelse(update & row$day <= old_lag & row$day > 1, }
     \CommentTok{#                  adjustLag(row$day, old_lag, newLag),}
     \CommentTok{#                  old_lag)}
\NormalTok{     newLag <-}\StringTok{ }\KeywordTok{ifelse}\NormalTok{(update }\OperatorTok{&}\StringTok{ }\NormalTok{row}\OperatorTok{$}\NormalTok{day }\OperatorTok{>}\StringTok{ }\DecValTok{1}\NormalTok{, }
                      \KeywordTok{adjustLag}\NormalTok{(row}\OperatorTok{$}\NormalTok{day, old_lag, newLag, }\DataTypeTok{restartExp =}\NormalTok{ F),}
\NormalTok{                      old_lag)}
\NormalTok{    newMu <-}\StringTok{  }\KeywordTok{ifelse}\NormalTok{(update, }
                     \KeywordTok{muAtNewTemp}\NormalTok{(newT, growth_import}\OperatorTok{$}\NormalTok{mumax[allele_index]),}
\NormalTok{                     prev_row}\OperatorTok{$}\NormalTok{newMu)}
\NormalTok{  data}\OperatorTok{$}\NormalTok{newTemp[i] <-}\StringTok{ }\NormalTok{newT}
\NormalTok{  data}\OperatorTok{$}\NormalTok{newLag[i] <-}\StringTok{ }\NormalTok{newLag}
\NormalTok{  data}\OperatorTok{$}\NormalTok{newMu[i] <-}\StringTok{ }\NormalTok{newMu}
  
  \CommentTok{#Calculate the log10N count using our new growth parameters}
\NormalTok{  newCount <-}\StringTok{ }\KeywordTok{log10N_func}\NormalTok{(row}\OperatorTok{$}\NormalTok{day, newLag, newMu,data}\OperatorTok{$}\NormalTok{logMPN_init[i],growth_import}\OperatorTok{$}\NormalTok{LOG10Nmax[allele_index])}
\NormalTok{  oldCount <-}\StringTok{ }\KeywordTok{log10N_func}\NormalTok{(row}\OperatorTok{$}\NormalTok{day}\DecValTok{-1}\NormalTok{, newLag, newMu,data}\OperatorTok{$}\NormalTok{logMPN_init[i],growth_import}\OperatorTok{$}\NormalTok{LOG10Nmax[allele_index])}
\NormalTok{  data}\OperatorTok{$}\NormalTok{count[i] <-}\StringTok{ }\KeywordTok{ifelse}\NormalTok{(row}\OperatorTok{$}\NormalTok{day}\OperatorTok{==}\DecValTok{1}\NormalTok{, newCount,}
\NormalTok{                          prev_row}\OperatorTok{$}\NormalTok{count }\OperatorTok{+}\StringTok{ }\NormalTok{(newCount}\OperatorTok{-}\NormalTok{oldCount))}
\NormalTok{\}}
\end{Highlighting}
\end{Shaded}


\end{document}
